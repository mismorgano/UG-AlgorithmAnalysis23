\documentclass[12pt]{article}
\usepackage{mathtools}
\usepackage[spanish]{babel}
\usepackage[a4paper, total={6in, 8in}]{geometry}
\usepackage{parskip}

\title{Análisis de Algoritmos y Matemáticas Discretas \\
    Tarea 3 - Divide y Vencerás }

\author{Antonio Barragan Romero}

\begin{document}
\maketitle
    \subsection*{Analisis de complejidad de : \emph{Sumas Parciales - Divide y Vencerás}}

    \textbf{Solución:}

    Sea $T(n)$ la función que mide el tiempo de ejecución de \verb|subarrays_sum_less_than|
    Podemo notar que Dividir el problem solo es calcular el punto medio del arreglo, lo cual se 
    hace en tiempo constante.
    Luego de Manera recursiva resolvemos dos problemas de tamaño $n/2$, lo cual contribuye en 
    $2T(n/2)$ al tiempo de ejecución.
    Analicemos pues el tiempo de ejecución de \verb|merge|.
    Lo primero que hace es crear un \verb|ordered_set| y recorriendo de la mitad del arreglo al incio agrega
    las sumas parciales (de izquierda a derecha).
    Vimos en clase que insertar es $O(\log(n))$ y como recorremos la mitad, pues hacemos $n/2$ inserciones, 
    entonces el tiempo de ejecución de llenar el \verb|ordered_set| es $O(n\log(n))$.
    Despues se recorre de la mitad al final del arreglo calculando sus sumas parciales, por cada suma $r$, 
    buscamos en el \verb|ordered_set| el numero de subarrays cuya suma es menor a $t-r$, para ello 
    llamamos a \verb|order_of_key|, por propiedades del \verb|order_of_key| esto se hace en $O(\log(n))$ 
    y como hacemos $n/2$ busquedas tenemos un tiempo de ejecución de $O(n\log(n))$.

    Lo anterior nos dice que la complejidad del \verb|merge| es $O(n\log(n))$.
    Se sigue que 
    \[
        T(n) = \begin{cases}
            k & \text{ para }n \le 2,\\
            2T(n/2) + O(n\log(n)) & \text{ para } n>2.
        \end{cases}
    \]
    Se puede ver que lo anterior no entra en ningun caso del Teorema Maestro, por lo cual calculemos 
    $T(n)$ y para facilitar los calculos supongamos que $n=2^p$, entonces tenemos que $\log(n) = p$, 
    por lo cual 
    \begin{align*}
        T(n) = 2T(2^p) &\le 2T(2^{p-1}) + c 2^{p}p \\
             &\le 2\left[ 2T(2^{p-2}) + c 2^{p-1}(p-1)\right] + c 2^{p}p \\
             &= 2^2T(2^{p-2}) + c2^p((p-1) + p)\\
             &\le 2^2\left[ 2T(2^{p-3}) + c 2^{p-2}(p-2)\right] + + c2^p((p-1) + p)\\
             &= 2^3T(2^{p-3}) + c2^p \left( (p-2) + (p-1) + p\right) \\
             &\vdots\\
             &\le 2^pT(1) + c2^p\sum_{i=1}^{p} i \\
             &= nk + c n \frac{\log(n)(\log(n) + 1)}{2} \\
             &= nk + cn\frac{\log(n)^2}{2} + cn\frac{\log(n)}{2}\\
             &= O(n\log^2(n)).
    \end{align*}
    De lo anterior tenemos que $T(n) = O(n\log^2(n))$.
\end{document}